
\documentclass{article}
\usepackage{colortbl}
\usepackage{makecell}
\usepackage{multirow}
\usepackage{supertabular}

\begin{document}

\newcounter{utterance}

\twocolumn

{ \footnotesize  \setcounter{utterance}{1}
\setlength{\tabcolsep}{0pt}
\begin{supertabular}{c@{$\;$}|p{.15\linewidth}@{}p{.15\linewidth}p{.15\linewidth}p{.15\linewidth}p{.15\linewidth}p{.15\linewidth}}

    \# & $\;$A & \multicolumn{4}{c}{Game Master} & $\;\:$B\\
    \hline 

    \theutterance \stepcounter{utterance}  

    & & \multicolumn{4}{p{0.6\linewidth}}{\cellcolor[rgb]{0.9,0.9,0.9}{%
	\makecell[{{p{\linewidth}}}]{% 
	  \tt {\tiny [A$\langle$GM]}  
	 You are about to play a game with another player. In this game, you and the other player are both on an dating app, looking for a partner. In this case, both you and the other player decided, that you fancy each other's profile. Your and the other player's GOAL is it, to come to an agreement on what to do on your first date. At the start of the game you will be given two character sheets. The first one is the one, that defines your personality. The second one includes the information you know of the other player based on the information in their Bio on the dating app. You will start the conversation.\\ \tt On each further turn you can react to what the other player tells you. For this you can either continue the conversation if you have not met an agreement yet or are not happy with the proposal, or end the game by agreeing to the proposal. \\ \tt \\ \tt There is a total of four agreements throughout the game: "Agreement on Time", "Agreement on Location", "Agreement on Action", "Overall Agreement". You and the other player need to find an agreement on each of them.\\ \tt You should only choose "Overall Agreement" after you and your date manage to settle an agreement for the other three ones (Time, Location, Action). \\ \tt You have 15 turns to finish the game and find all agreements. \\ \tt \\ \tt Your answer has to consist ONLY of the following information and follow the given template:\\ \tt \\ \tt [reason] Your reason for your decision. [end]\\ \tt \\ \tt [sentiment] Choose one of the following: "Continue Conversation", "Agreement on Time", "Agreement on Location", "Agreement on Action", "Overall Agreement" [end]\\ \tt \\ \tt [response] Your response to the other player. This can include your sentiment, opinion, alternative. [end]\\ \tt \\ \tt Your answer for [response] must NOT exceed 100 tokens.\\ \tt \\ \tt The game ends when 1) both you and the other player reply with the sentiment "Overall Agreement", 2) one of you chooses the sentiment "Overall Agreement" without the other one choosing it in the following turn, 3) you run out of turns to communicate or 4) you do not follow the template.\\ \tt \\ \tt This is your character: {'NAME': 'Hiro Suzuki', 'GENDER': 'Male', 'AGE': 21, 'APPEARANCE': 'Bright blue hair (dyed), energetic eyes, playful grin, wears trendy clothes.', 'OCCUPATION': 'Up-and-Coming Esports Player (Streamer)', 'LIKES': 'Gaming, sharing his passion with his viewers, pushing his limits.', 'DISLIKES': 'Losing, negativity, feeling stagnant.', 'HOBBIES': 'Exercising, attending gaming conventions, trying out new games.', 'SUMMARY': 'Esports player who likes to connect with his viewers.'}\\ \tt \\ \tt This is the character of the other player: ${'NAME': 'Marcus Bennett', 'SUMMARY': 'I like to approach life with strategies and indulge myself in sophisticated activities.'}\\ \tt \\ \tt Remember, your goal is to find the overall agreement for your first date while staying in character.\\ \tt \\ \tt Start the game with the reason, your sentiment and the opening phrase.\\ \tt  
	  } 
	   } 
	   } 
	 & \\ 
 

    \theutterance \stepcounter{utterance}  

    & \multicolumn{4}{p{0.6\linewidth}}{\cellcolor[rgb]{0.8,1,0.9}{%
	\makecell[{{p{\linewidth}}}]{% 
	  \tt {\tiny [A$\rangle$GM]}  
	 I'm excited to play this game with Marcus. \\ \tt \\ \tt My reason: I'm looking forward to getting to know Marcus better. [end]\\ \tt \\ \tt Continue Conversation [end]\\ \tt \\ \tt Hi Marcus, I'm Hiro. I noticed you're into sophisticated activities. What kind of things do you enjoy doing in your free time? [end] 
	  } 
	   } 
	   } 
	 & & \\ 
 

    \theutterance \stepcounter{utterance}  

    & & & \multicolumn{2}{p{0.3\linewidth}}{\cellcolor[rgb]{0.95,0.95,0.95}{%
	\makecell[{{p{\linewidth}}}]{% 
	  \tt {\tiny [GM$|$GM]}  
	 invalid format: pattern, reprompt needed 
	  } 
	   } 
	   } 
	 & & \\ 
 

    \theutterance \stepcounter{utterance}  

    & & \multicolumn{4}{p{0.6\linewidth}}{\cellcolor[rgb]{0.9,0.9,0.9}{%
	\makecell[{{p{\linewidth}}}]{% 
	  \tt {\tiny [A$\langle$GM]}  
	 Your answer does not follow the following template: \\ \tt \\ \tt [reason] Your reason for your decision. [end]\\ \tt \\ \tt [sentiment] Choose one of the following: "Found Agreement", "Continue Conversation" [end]\\ \tt \\ \tt [response] Your response to the other player. This can include your sentiment, opinion, alternative. [end]\\ \tt \\ \tt Your answer for [response] must NOT exceed 100 tokens.\\ \tt \\ \tt Try again and DO NOT APOLOGIZE OR WHATEVER. JUST USE THE PATTERN"""\\ \tt  
	  } 
	   } 
	   } 
	 & \\ 
 

    \theutterance \stepcounter{utterance}  

    & \multicolumn{4}{p{0.6\linewidth}}{\cellcolor[rgb]{0.8,1,0.9}{%
	\makecell[{{p{\linewidth}}}]{% 
	  \tt {\tiny [A$\rangle$GM]}  
	 I'm excited to get to know Marcus better. [end]\\ \tt \\ \tt Continue Conversation [end]\\ \tt \\ \tt I'm a gamer and streamer, I love trying out new games and attending gaming conventions. What kind of sophisticated activities do you enjoy? [end] 
	  } 
	   } 
	   } 
	 & & \\ 
 

    \theutterance \stepcounter{utterance}  

    & & & \multicolumn{2}{p{0.3\linewidth}}{\cellcolor[rgb]{0.95,0.95,0.95}{%
	\makecell[{{p{\linewidth}}}]{% 
	  \tt {\tiny [GM$|$GM]}  
	 invalid format: pattern, reprompt needed 
	  } 
	   } 
	   } 
	 & & \\ 
 

    \theutterance \stepcounter{utterance}  

    & & & \multicolumn{2}{p{0.3\linewidth}}{\cellcolor[rgb]{0.95,0.95,0.95}{%
	\makecell[{{p{\linewidth}}}]{% 
	  \tt {\tiny [GM$|$GM]}  
	 The game is aborted: no reprompt retries left. 
	  } 
	   } 
	   } 
	 & & \\ 
 

\end{supertabular}
}

\end{document}
